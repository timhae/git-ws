\documentclass[aspectratio=169]{beamer}
% \includeonlyframes{schluss20,schluss30}
\synctex=1
\usepackage[utf8]{inputenc}
\usepackage[ngerman]{babel}
\usepackage[T1]{fontenc}
\usepackage{graphicx}
\usepackage{verbatim}
\usepackage{csquotes}
\usepackage{eurosym}
\usepackage{wasysym}
\usepackage{amssymb}
\MakeAutoQuote{„}{”}
\usepackage[absolute,overlay]{textpos} % for \textblock

\usepackage{fancyvrb}
\usepackage{newverbs}
\usepackage{xcolor}

\usepackage{tikz}
\usetikzlibrary{%
  positioning,
  calc,
}

\usepackage{gitdags}
% Zum Zeichnen von git-Graphen
% https://tex.stackexchange.com/a/198345/2381
% hängt ab von xcolor-solarized
% Zur Vermeidung von Komplierungsschwierigkeiten liegen xcolor-solarized.sty
% und gitdags.sty direkt im Hauptverzeichnis

\usepackage{lmodern}


\hypersetup{
  colorlinks=true,
  linkcolor=, %
  urlcolor=[rgb]{.2, .2, .5} %
  % allcolors=[rgb]{0, 0, 0} % schwarz
}


% boilerplate stuff to get colred background of \verb|...|
% quelle: https://tex.stackexchange.com/a/141128/2381
\definecolor{cverbbg}{gray}{0.93}

\newenvironment{cverbatim}
 {\SaveVerbatim{cverb}}
 {\endSaveVerbatim
  \flushleft\fboxrule=0pt\fboxsep=.5em
  \colorbox{cverbbg}{\BUseVerbatim{cverb}}%
  \endflushleft
}

\newcommand{\ctexttt}[1]{\colorbox{cverbbg}{\texttt{#1}}}
\newverbcommand{\cverb}
  {\setbox\verbbox\hbox\bgroup}
  {\egroup\colorbox{cverbbg}{\box\verbbox}}

% end of colored verb boilerplate


\usetheme{Dresden}
\setbeamertemplate{navigation symbols}{}
\setbeamertemplate{footline}[frame number]

\title{Versionsverwaltung mit git: Warum und wie.}
\subtitle{Erstsemester-Einführung Informatik, 22.10.2020}

\begin{document}

\begin{frame}
  \begin{center}%
    \includegraphics[width=3cm]{img-src/fsfw-logo-with-text.pdf}\\

    \vspace*{-0.5\baselineskip}

    \parbox{.95\columnwidth}{\centering\Large\inserttitle}

    \vspace*{\baselineskip}

    \structure{\large \insertsubtitle}
  \end{center}
\end{frame}

%%%%%%%%%%%%%%%%%%%%%%%%%%%%%%%%%%%%%%%%%%%%%%%%%%%%%%%%%%%%%%%%%%%%%%%%%%%%%%%%

\begin{frame}[label=ct4]{FSFW}

  \begin{center}
    \url{https://fsfw-dresden.de/}
    $\;\;\left\{\;\;\text{
        \parbox{2.3cm}{
          \texttt{uni-stick}\\[1mm]
          \texttt{mitmachen}\\[1mm]
          \texttt{ringvorlesung}\\[2mm]
          \texttt{\textbf{git-ws}}\\[-3mm]
        }}
    \right.$

    \vspace*{2\bigskipamount}

    \includegraphics[width=50mm]{img-src/fsfw-netzwerke}
  \end{center}

  \begin{itemize}
  \item Hochschulgruppe an der TU, Studierende und andere Leute
  \item Bisherige Projekte:
    \begin{itemize}
    \item Linux-Install-Party, Linux-Presentation-Day
    \item Sprechstunde zu \LaTeX{} u.a.
    \item „Uni-Stick” mit freier Software
    \item Ringvorlesung: Freie Software und Freies Wissen als Beruf
    \end{itemize}

  \end{itemize}

\end{frame}

%%%%%%%%%%%%%%%%%%%%%%%%%%%%%%%%%%%%%%%%%%%%%%%%%%%%%%%%%%%%%%%%%%%%%%%%%%%%%%%%

\begin{frame}[label=ol1]
  \frametitle{Gliederung}
  \tableofcontents
\end{frame}

%%%%%%%%%%%%%%%%%%%%%%%%%%%%%%%%%%%%%%%%%%%%%%%%%%%%%%%%%%%%%%%%%%%%%%%%%%%%%%%%
\section{Warum Versionsverwaltung?}

\begin{frame}[label=why10]
\frametitle{Warum Versionsverwaltung?}

\begin{itemize}
  \item Projekte bestehen aus schrittweisen Änderungen
  \item Bedürfnis, zu vorherigem Zustand zurückkehren zu können ("`Savegame"')
  \item Naiver Ansatz: \hspace{10mm}
  \raisebox{-0.5\height}{\includegraphics[width=40mm]{img-src/my-software-dirs}}

  \item Probleme:
  \begin{itemize}
   \item Speicherplatz
   \item Fehlende Übersicht
   \item Skaliert nicht (Teamwork)
  \end{itemize}

  \end{itemize}
\end{frame}

%%%%%%%%%%%%%%%%%%%%%%%%%%%%%%%%%%%%%%%%%%%%%%%%%%%%%%%%%%%%%%%%%%%%%%%%%%%%%%%%
% TODO: image source http://techidiocy.com/understand-git-clone-command-svn-checkout-vs-git-clone/
% https://softwareengineering.stackexchange.com/questions/136079/are-there-any-statistics-that-show-the-popularity-of-git-versus-svn
% http://www.wastedpotential.com/version-control-throwdown-git-vs-svn/
\section{Warum Git?}
\begin{frame}[label=why20]
\frametitle{Warum Git? (1)}

\begin{itemize}
  \item Lösung 1: zentrale Versionsverwaltung
  \begin{itemize}
   \item CVS (1986), SVN (2000)
   \item Idee: Zentrales Repositorium und Arbeitskopien \\[1ex]
       \hspace*{1cm}\begin{tikzpicture}[%
           node distance=2cm and 1cm,
           location/.style={
             % The shape:
             rectangle,
             % The size:
             minimum size=6mm,
             % The border:
             very thick,
             draw=red!50!black!50,
             % 50% red and 50% black,
             % and that mixed with 50% white
             % The filling:
             top color=white,
             % a shading that is white at the top...
             bottom color=red!50!black!20, % and something else at the bottom
           },
           checkout/.style={
             <->,
             shorten <=2pt,
             shorten >=2pt,
             thick,
             color=black!40!white,
           }]

         ]
         \node[location, draw=green!50!black!50, bottom color=green!50!black!20] (repository) {Zentrales Repo};
         \node[location] (user1) [below left of=repository] {Arbeitskopie 1};
         \node[location] (user2) [below of=repository, yshift=-1ex] {Arbeitskopie 2};
         \node[location] (user3) [below right of=repository] {Arbeitskopie 3};

         \draw[checkout] ($ (repository.south) + (-2pt, 0) $) -- (user1.north);
         \draw[checkout] (repository.south) -- (user2.north);
         \draw[checkout] ($ (repository.south) + (2pt, 0) $) -- (user3.north);
       \end{tikzpicture}
   \end{itemize}
  \item Probleme:
  \begin{itemize}
   \item Abhängig von Server-Erreichbarkeit
   \item Performanz
  \end{itemize}

  \end{itemize}
\end{frame}

%%%%%%%%%%%%%%%%%%%%%%%%%%%%%%%%%%%%%%%%%%%%%%%%%%%%%%%%%%%%%%%%%%%%%%%%%%%%%%%%
\begin{frame}[label=why30]
\frametitle{Warum Git? (2)}

\begin{itemize}
  \item Lösung 2: \textbf{de}zentrale Versionsverwaltung
  \begin{itemize}
   \item mercurial (2005), bazaar (2005), \textbf{git} (2005)
   \item Idee: Jeder hat ein vollwertiges Repositorium \\[1ex]
      \hspace*{1cm}\begin{tikzpicture}[%
           node distance=2cm and 1cm,
           location/.style={
             % The shape:
             rectangle,
             % The size:
             minimum size=6mm,
             % The border:
             very thick,
             draw=green!50!black!50,
             % 50% red and 50% black,
             % and that mixed with 50% white
             % The filling:
             top color=white,
             % a shading that is white at the top...
             bottom color=green!50!black!20, % and something else at the bottom
           },
           checkout/.style={
             <->,
             shorten <=2pt,
             shorten >=2pt,
             thick,
             color=black!40!white,
           }]

         ]
         \node[location] (user1) {Git Repo 1};
         \node[location] (user2) [below  left of=user1] {Git Repo 2};
         \node[location] (user3) [below right of=user1] {Git Repo 3};

         \draw[checkout] ($ (user1.south) + (-2pt, 0) $) -- (user2.north);
         \draw[checkout] (user3.west) -- (user2.east);
         \draw[checkout] ($ (user1.south) + (2pt, 0) $) -- (user3.north);

     \end{tikzpicture}
  \end{itemize}
  \item Vorteile:
  \begin{itemize}
   \item Alle Operationen lokal $\rightarrow$ schnell, unabhängig
   \item Einfaches "`branching"' und "`merging"'
   \item ...
  \end{itemize}
  \end{itemize}
\end{frame}

%%%%%%%%%%%%%%%%%%%%%%%%%%%%%%%%%%%%%%%%%%%%%%%%%%%%%%%%%%%%%%%%%%%%%%%%%%%%%%%%
\section{Git Einführung (mit Praxis)}
\begin{frame}[label=ef1]
\frametitle{Einführung in git – Was ist ein Repo?}
\begin{columns}
  \column{0.245\textwidth}
  \begin{tikzpicture}
    \gitDAG[grow down sep = 1em]{
      A -- B -- C
    };
    \gitbranch{master}{left=of C}{C};
    \gitHEAD{above=of master}{master}
    \visible<4->{
    \SAandWT
    }
  \end{tikzpicture}
  \column{0.05\textwidth}
  ~
  \column{0.75\textwidth}
  \begin{itemize}
  \item Graph von Versionen einer Ordnerstruktur und deren Inhalt mit
    Metadaten (\emph{Commit-ID}, Autor, Beschreibungstext)
  \item Commit-ID abgeleitet aus dem Inhalt und dem Graphen
    (kryptographische Hash-Funktion)\\
    \pause
    {\tiny Beispiel: b52c95e791e1dac76b7f70292e366de7caa76178}
  \item \emph{\texttt{HEAD}}: Knoten im Graphen; momentaner
    Bezugspunkt für Operationen
  \item \emph{refs}: referenzieren Knoten im Graphen (Beispiele:
    \texttt{HEAD}, \texttt{HEAD\^{}3}, \texttt{master}, \texttt{my\_branch}, \texttt{b52c95e} {\tiny (abgekürzte
    Commit-ID)})
  \end{itemize}
\end{columns}

\begin{textblock*}{40mm}[0.,0.](30mm,27mm)
{\color<3->{gray}{\tiny
Start\\[7.5mm]
Nachfolger\\[6.5mm]
Nach-Nachfolger
}
}
\end{textblock*}

\end{frame}

\begin{frame}[label=gitintro10]
\frametitle{Einführung in git – Verwendung}
\begin{itemize}
 \item Wir empfehlen: git Bedienung via Kommandozeile
   %% NOTES:
   % \begin{itemize}
   % \item Alternative: Integration in die IDE
   % \item Kommandozeile kann mehr
   % \item[$\Rightarrow$] können hilft oft weiter
   % \end{itemize}
 \item Syntax: \texttt{git <command> [<args>]}
 \item Beispiele:

 \begin{itemize}
  \item \texttt{git init}
  \item \texttt{git add myscript.py}
  \item \texttt{git commit -m "{}add basic functionality"}
  \item \texttt{git push}\\[2mm]
  \pause
  \item \texttt{git status}
  \item \texttt{git log}
  \item \texttt{git branch develop}
  \item \texttt{git checkout master}
  \item \texttt{git merge develop}
  \item \texttt{git blame myscript.py}
  \item \texttt{git diff}
  \item \texttt{git difftool}

 \end{itemize}
\end{itemize}

\begin{textblock*}{\textwidth}[0.5,0.](140mm, 60mm)

\visible<3->{
\begin{itemize}
 \item[] ~
\begin{itemize}

  \item \texttt{git clone}
  \item \texttt{git help <command>}
  \item \texttt{git rebase} \\[2mm]
  \item \texttt{git config}

  \item \texttt{gitk}

\end{itemize}
\end{itemize}
}
\end{textblock*}

\end{frame}

%%%%%%%%%%%%%%%%%%%%%%%%%%%%%%%%%%%%%%%%%%%%%%%%%%%%%%%%%%%%%%%%%%%%%%%%%%%%%%%%

\begin{frame}[label=gitintro20]
\frametitle{Praxis 1: Erste Schritte}
\begin{itemize}
    \item Installation: \texttt{apt install git} oder \url{https://git-scm.com/download/win}
 \item Konfiguration anpassen
 \begin{itemize}
  \item \texttt{git config -{}-global user.email "foo@bar.de"}
  \item \texttt{git config -{}-global user.name "Your Name"}
  \item ...
 \end{itemize}

 \item Eigenes Repo \texttt{foo} erstellen
 \begin{itemize}
  \item \texttt{mkdir foo}
  \item \texttt{cd foo}
  \item \texttt{git init}
 \end{itemize}
 \item Oder bestehendes Repo klonen
 \begin{itemize}
  \item \texttt{git clone <url>}
 \end{itemize}
 \pause
 \item Hintergrund: Wo speichert git\\ die relevanten Informationen?
 \item[$\rightarrow$] Verstecktes Verzeichnis \texttt{.git}



\end{itemize}
\begin{textblock*}{40mm}[0.,0.](70mm,70mm)
\visible<2->{
 \includegraphics[width=35mm]{img-src/git-dir1}
}
\end{textblock*}
\end{frame}
%%%%%%%%%%%%%%%%%%%%%%%%%%%%%%%%%%%%%%%%%%%%%%%%%%%%%%%%%%%%%%%%%%%%%%%%%%%%%%%%

\begin{frame}
  \frametitle{Theorie: typischer Ablauf / "`staging area"' (1)}
  \begin{center}
    %% Gehirnvirus zuerst und allein auf einer Folie
    %% Details mit schema auf einer späteren folie
    \includegraphics[width=.8\textwidth]{img-src/git-staging-area}
  \end{center}
\end{frame}

\begin{frame}[label=gitintro30]
\frametitle{Theorie: typischer Ablauf / "`staging area"' (2)}
\vspace{-8mm}
\begin{center}
  \begin{tikzpicture}[
    every text node part/.style={align=center},
    node distance=4.0cm,
    location/.style={
      % The shape:
      rectangle,
      % The size:
      minimum size=6mm,
      % The border:
      very thick,
      draw=red!50!black!50,
      % 50% red and 50% black,
      % and that mixed with 50% white
      % The filling:
      top color=white,
      % a shading that is white at the top...
      bottom color=red!50!black!20, % and something else at the bottom
      text height=4ex,
      text depth=0.25ex,
    }]

    \node[location] (untracked) {working \\ tree};
    \node[location] (staged) [right of=untracked] {index \\[-0.7em]};
    \node[location] (committed) [right of=staged] {local \\ repository};
    \node[location] (published) [right of=committed] {remote \\ repository};

    \draw[->, thick, color=black!40!white] (untracked.east) -- (staged.west)
          node [midway, above=0.0ex, color=black] {\texttt{git add}};
    \draw[->, thick, color=black!40!white] (staged.east) -- (committed.west)
          node [midway, above=0.0ex, color=black] {\texttt{git commit}};
    \draw[->, thick, color=black!40!white] (committed.east) -- (published.west)
          node [midway, above=0.0ex, color=black] {\texttt{git push}};
  \end{tikzpicture}
\end{center}
Wozu zweiphasiger Commit-Prozess?
\begin{itemize}
 \item Ermöglicht präzise, hoch aufgelöste Commits\\
 \begin{itemize}
  \item Änderungen mancher Dateien (\texttt{git add dir1/*.html})
  \item Nur bestimmte Änderungen einer Datei (\texttt{git add -p})
  \item Alle Änderungen übernehmen und comitten (\texttt{git commit -a})
 \end{itemize}
 \item[$\Rightarrow$] nachvollziehbare, aussagekräftige Commit-History
\end{itemize}


\end{frame}

%%%%%%%%%%%%%%%%%%%%%%%%%%%%%%%%%%%%%%%%%%%%%%%%%%%%%%%%%%%%%%%%%%%%%%%%%%%%%%%%
\begin{frame}[fragile,label=gitintro40]
\frametitle{Praxis: minimales Repo}
\vspace{-5mm}
\begin{columns}
% Trick: durch Einfügen einer weiteren Spalte wird die 1. Spalte nach links gedrückt
\column[t]{1.0\textwidth}
\begin{itemize}
 \item Inhalt erzeugen
\begin{itemize}
 \item \verb|printf "Hallo\nWelt\n" > README.md|
 \item \verb|git status|
 \item \verb|git add README.md| \qquad \visible<1->{{\scriptsize Tipp: Auto-Vervollständigung mit TAB}}
 \item \verb|git status|
 \item \verb|git commit -m "New content of README"| %\quad \visible<4->{{\scriptsize Fehler absichtlich}}
 \item \verb|git status|
\end{itemize}
\pause
 \item Änderungen durchführen, anzeigen und committen
\begin{itemize}
 \item \verb|sed -i -- "s/Welt/Leute/g" README.md|
 \item \verb|git diff|
 \pause
 \pause
 \item \verb|git commit -am "change Hello-message"|
\end{itemize}
 \pause
 \item Sich Überblick verschaffen
 \begin{itemize}
 \item \verb|git status|
 \item \verb|git log|
 \item \verb|gitk|
 \end{itemize}
%  \pause
%  \item Letzten commit korrigieren (mit Editor, default: \texttt{\$EDITOR})
%  \begin{itemize}
%  \item  \texttt{git config -{}-global core.editor mcedit}
%  \item \texttt{git commit -{}-amend}
%  \end{itemize}
\end{itemize}

\column[t]{0.1\textwidth}
~
\end{columns}


\begin{textblock*}{1cm}[0.,0.](95mm,50mm)
\visible<3->{
\includegraphics[height=25mm]{img-src/git-diff-snapshot}
}
\end{textblock*}

\end{frame}

%%%%%%%%%%%%%%%%%%%%%%%%%%%%%%%%%%%%%%%%%%%%%%%%%%%%%%%%%%%%%%%%%%%%%%%%%%%%%%%%
\begin{frame}[fragile,label=branches10]
\frametitle{Theorie: Branches}
\begin{columns}
\column{0.349\textwidth}
\only<1,2>{
\begin{tikzpicture}
      \useasboundingbox (-1.5,-4.5) rectangle +(4,4.5);

      % Commit DAG
      \gitDAG[grow down sep = 1em]{
        A -- B -- C
      };
      % Branch

      \gitbranch
        {master}     % node name and text
        {left=of C} % node placement
        {C}          % target
      % HEAD reference
      \gitHEAD
        {above=of master} % node placement
        {master}          % target
    \end{tikzpicture}
}
\only<3,5>{
\begin{tikzpicture}
     \useasboundingbox (-1.5,-4.5) rectangle +(4,4.5);

      % Commit DAG
      \gitDAG[grow down sep = 1em]{
        A -- B --{
              C,
              D -- E,
        }
      };
      % Branch

      \gitbranch
        {master}     % node name and text
        {left=of C} % node placement
        {C}          % target
      % HEAD reference
      \gitHEAD
        {above=of master} % node placement
        {master}          % target

      \gitbranch
        {bugfix}     % node name and text
        {left=of E} % node placement
        {E}
    \end{tikzpicture}
} % end of only
\only<4,6>{
\begin{tikzpicture}
      \useasboundingbox (-1.5,-4.5) rectangle +(4,4.5);

      % Commit DAG
      \gitDAG[grow down sep = 1em]{
        A -- B --{
              C,
              D -- E,
        }
      };

      % Branch
      \gitbranch
        {master}     % node name and text
        {left=of C} % node placement
        {C}          % target
      % HEAD reference
      \gitHEAD
        {below=of bugfix} % node placement
        {bugfix}          % target

      \gitbranch
        {bugfix}     % node name and text
        {left=of E} % node placement
        {E}
    \end{tikzpicture}
} % end of only
\only<7>{
\begin{tikzpicture}
      \useasboundingbox (-1.5,-4.5) rectangle +(4,4.5);

      % Commit DAG
      \gitDAG[grow down sep = 1em]{
        A -- B --{
              C,
              D -- E,
        }
      };

      % Branch
      \gitbranch
        {master}     % node name and text
        {left=of C} % node placement
        {C}          % target
      \gitbranch
        {bugfix}     % node name and text
        {left=of E} % node placement
        {E}

      \gitHEAD
        {above=of D, xshift=1.5em} % node placement
        {D}
    \end{tikzpicture}
} % end of only
\only<8>{
\begin{tikzpicture}
      \useasboundingbox (-1.5,-4.5) rectangle +(4,4.5);

      % Commit DAG
      \gitDAG[grow down sep = 1em]{
        A -- B --{
              C,
              D -- E,
        }
      };

      % Branch
      \gitbranch
        {master}     % node name and text
        {left=of C} % node placement
        {C}          % target
      \gitbranch
        {bugfix}     % node name and text
        {left=of E} % node placement
        {E}

      \gitbranch
        {experiment}     % node name and text
        {above=of D, xshift=1.5em} % node placement
        {D}
      % HEAD reference
      \gitHEAD
        {above=of experiment} % node placement
        {experiment}          % target

    \end{tikzpicture}
} % end of only
\only<9>{
\begin{tikzpicture}
      \useasboundingbox (-1.5,-4.5) rectangle +(4,4.5);

      % Commit DAG
      \gitDAG[grow down sep = 1em]{
        A -- B --{
              C,
              D --{ E,
                    F -- G,
                  }
        }
      };


      % Branch
      \gitbranch
        {master}     % node name and text
        {left=of C} % node placement
        {C}          % target
      \gitbranch
        {bugfix}     % node name and text
        {left=of E} % node placement
        {E}

      \gitbranch
        {experiment}     % node name and text
        {left=of G} % node placement
        {G}
      % HEAD reference
      \gitHEAD
        {left=of experiment} % node placement
        {experiment}          % target

    \end{tikzpicture}
} % end of only
\column{0.65\textwidth}
\begin{itemize}
\item Unkompliziertes paralleles Arbeiten an verschieden Versionen
\item<2-> Der aktive Branch folgt \texttt{HEAD}
\item<3-> beliebig viele Branches möglich
\item<4-> Branch/Revision wechseln:\\
% !! aus unklarem Grund funktioniert das hier nicht (\cverb oder \verb):
% \only<4,6>{\verb|git checkout bugfix|%}
% \only<5>{\verb|git checkout master|}
% \only<7->{\verb|git checkout <ref>|}

% deswegen nehmen wir jetzt \texttt (ohne Farbe)
\only<4,6>{\texttt{git checkout bugfix}}
\only<5>{\texttt{git checkout master}}
\only<7->{\texttt{git checkout <ref>}}
\item<8-> neuer Branch auf \texttt{HEAD} erstellen:\\
\verb|git checkout -b experiment|
\end{itemize}
\end{columns}
\end{frame}

%%%%%%%%%%%%%%%%%%%%%%%%%%%%%%%%%%%%%%%%%%%%%%%%%%%%%%%%%%%%%%%%%%%%%%%%%%%%%%%%

\begin{frame}[fragile,label=branches20]
\frametitle{Theorie: Zusammenfassung Branches}
\begin{columns}
  \column{0.349\textwidth}
\begin{tikzpicture}
      \useasboundingbox (-1.5,-4.5) rectangle +(4,4.5);

      % Commit DAG
      \gitDAG[grow down sep = 1em]{
        A -- B --{
              C,
              D --{ E,
                    F -- G,
                  }
        }
      };

      % Branch
      \gitbranch
        {master}     % node name and text
        {left=of C} % node placement
        {C}          % target
      \gitbranch
        {bugfix}     % node name and text
        {left=of E} % node placement
        {E}

      \gitbranch
        {experiment}     % node name and text
        {left=of G} % node placement
        {G}
      % HEAD reference
      \gitHEAD
        {left=of experiment} % node placement
        {experiment}          % target

    \end{tikzpicture}

  \column{0.65\textwidth}
  \begin{center}
    \Large Branches sind \emph{lokale} Lesezeichen auf Knoten im
    Revisionsgraphen. Beim Anlegen eines neuen Commits folgt der aktive
    Branch dem neuen \texttt{HEAD}.
  \end{center}
\end{columns}
\end{frame}


\begin{frame}[fragile,label=branches30]
\frametitle{Theorie: Mergen – Zusammenführen von Zweigen}
\begin{columns}
\column{0.36\textwidth}
\only<1>{
\begin{tikzpicture}
      \useasboundingbox (-1.5,-4.5) rectangle +(4,4.5);

      % Commit DAG
      \gitDAG[grow down sep = 1em]{
        A -- B -- C  -- D -- E

      };

        \gitbranch
        {master}     % node name and text
        {left=of B} % node placement
        {B}

        \gitbranch
        {bugfix}     % node name and text
        {left=of E} % node placement
        {E}
      % HEAD reference
      \gitHEAD
        {below=of bugfix} % node placement
        {bugfix}
\end{tikzpicture}
}%end of only
\only<2>{
\begin{tikzpicture}
      \useasboundingbox (-1.5,-4.5) rectangle +(4,4.5);

      % Commit DAG
      \gitDAG[grow down sep = 1em]{
        A -- B -- C  -- D -- E

      };

        \gitbranch
        {master}     % node name and text
        {left=of B} % node placement
        {B}

        \gitbranch
        {bugfix}     % node name and text
        {left=of E} % node placement
        {E}
      % HEAD reference
      \gitHEAD
        {below=of master} % node placement
        {master}
\end{tikzpicture}
}%end of only
\only<3>{
\begin{tikzpicture}
      \useasboundingbox (-1.5,-4.5) rectangle +(4,4.5);

      % Commit DAG
      \gitDAG[grow down sep = 1em]{
        A -- B -- C  -- D -- E

      };

        \gitbranch
        {bugfix}     % node name and text
        {left=of E} % node placement
        {E}

        \gitbranch
        {master}     % node name and text
        {right=of E} % node placement
        {E}

        % HEAD reference
      \gitHEAD
        {below=of master} % node placement
        {master}
\end{tikzpicture}
}%end of only

% ---

\only<4>{
\begin{tikzpicture}
      \useasboundingbox (-1.5,-4.5) rectangle +(4,4.5);

      % Commit DAG
      \gitDAG[grow down sep = 1em]{
        A -- B --{
              C  -- D,
              E -- F,
              }
      };

      \gitbranch
        {master}     % node name and text
        {left=of D} % node placement
        {D}

   \gitbranch
        {bugfix}     % node name and text
        {right=of F} % node placement
        {F}

      \gitHEAD
        {below=of master} % node placement
        {master}
\end{tikzpicture}
}%end of only
\only<5->{
\begin{tikzpicture}
      \useasboundingbox (-1.5,-4.5) rectangle +(4,4.5);

      % Commit DAG
      \gitDAG[grow down sep = 1em]{
        A -- B --{
              C  -- D,
              E -- F,
              } -- H
      };

      \gitbranch
        {master}     % node name and text
        {left=of H} % node placement
        {H}

   \gitbranch
        {bugfix}     % node name and text
        {right=of F} % node placement
        {F}

      \gitHEAD
        {below=of master} % node placement
        {master}
\end{tikzpicture}
}%end of only

\column{0.72\textwidth}
\begin{itemize}
\item Fall 1: Fast-Forward
\pause
\begin{itemize}
 \item \verb|git checkout master|
 \pause
 \item \verb|git merge bugfix|
\end{itemize}
\pause
\bigskip
\item Fall 2: Parallele Zweige
\begin{itemize}
 \item \verb|git checkout master|
 \pause
 \item \verb|git merge bugfix|
 \item[$\Rightarrow$] Erzeugung eines "`Merge-Commits"'
 \pause
 \item Automatische Konfliktlösung ziemlich gut
 \item Gelegentlich manueller Eingriff notwendig
 \end{itemize}
\end{itemize}
\end{columns}

\end{frame}
%%%%%%%%%%%%%%%%%%%%%%%%%%%%%%%%%%%%%%%%%%%%%%%%%%%%%%%%%%%%%%%%%%%%%%%%%%%%%%%%
\begin{frame}[fragile,label=branches40]
  \frametitle{Theorie: Mergen – Konflikte auflösen}
  \begin{itemize}
  \item Konflikte beim Mergen: beide Versionen werden in der Datei markiert eingefügt\\
    \hspace*{2em}\texttt{Gleiche Zeilen 1,} \\
    \hspace*{2em}\texttt{<{}<{}<{}<{}<{}<{}< HEAD} \\
    \hspace*{2em}\texttt{in unserem Zweig geänderte Zeilen,} \\
    \hspace*{2em}\texttt{=======} \\
    \hspace*{2em}\texttt{im anderen Zweig geänderte Zeile,} \\
    \hspace*{2em}\texttt{>{}>{}>{}>{}>{}>{}> other-branch} \\
    \hspace*{2em}\texttt{Gleiche Zeilen 2}
  \item manuell editieren um den Konflikt aufzuheben (z.\,B. beide
    Zeilen behalten, die Änderungen in beiden Zeilen zusammenführen,
    eine Version behalten), die Marker entfernen
  \item \verb|git add <conflicting-file>|
  \item \verb|git commit|
  \end{itemize}
\end{frame}


%%%%%%%%%%%%%%%%%%%%%%%%%%%%%%%%%%%%%%%%%%%%%%%%%%%%%%%%%%%%%%%%%%%%%%%%%%%%%%%%
\newcounter{taskcounter}

\begin{frame}[fragile,label=uebung10]
\frametitle{Praxis}
\begin{itemize}
 \item Bereitgestelltes nicht-triviales Repo:\\
%  {\scriptsize \url{https://pubgit.sotecware.net/fsfw-git-workshop/lyrik}}
 {\scriptsize \url{https://github.com/fsfw-dresden/git-ws-lyrik.git}}
 \item Basis für verschiedene Aufgaben (Anregungen zum spielen)
\begin{enumerate}
 \item Repo klonen \verb|git clone <url>|
 \item Überblick verschaffen: \verb|gitk --all|
 \begin{enumerate}[a)]
  \item Wie viele Commits, Committer gibt es?
  \item Wie viele Branches?
 \end{enumerate}
 \item Änderungen vornehmen
 \begin{enumerate}[a)]
 \item zum Branch \textit{weimar} wechseln
 \item in Datei \verb|gedichte/prometheus.md| 'YYY' durch 'ich' ersetzen,
 \item committen
 \item analog im Branch \textit{london} \verb|sonnets/text1.md| 'XXX' durch 'thee' ersetzen
 \end{enumerate}
\setcounter{taskcounter}{\value{enumi}}
\end{enumerate}

\end{itemize}

\end{frame}


%%%%%%%%%%%%%%%%%%%%%%%%%%%%%%%%%%%%%%%%%%%%%%%%%%%%%%%%%%%%%%%%%%%%%%%%%%%%%%%%

\begin{frame}[fragile,label=uebung20]
\frametitle{Praxis (2)}
\begin{enumerate}
 \setcounter{enumi}{\value{taskcounter}}
 \item commit-History einzelner Dateien anzeigen
 \begin{enumerate}[a)]
  \item \verb|git blame AUTHORS.md|
  \item \verb|git blame sonnets/text1.md|
 \end{enumerate}

 \item Änderungen anzeigen
 \begin{enumerate}[a)]
  \item ... seit dem vorletzten Commit: \verb|git diff HEAD~|
  \item[$\rightarrow$] beliebige Änderungen vornehmen
  \item ... seit dem letzten Commit: \verb|git diff|
  \end{enumerate}
  \item Branch \textit{london} in \textit{master} mergen
  \begin{enumerate}[a)]
   \item master auschecken: \verb|git checkout master|
   \item merge durchführen: \verb|git merge london|
   \item Ergebnis anschauen: \verb|gitk --all|
  \end{enumerate}

 \setcounter{taskcounter}{\value{enumi}}
\end{enumerate}



\end{frame}


%%%%%%%%%%%%%%%%%%%%%%%%%%%%%%%%%%%%%%%%%%%%%%%%%%%%%%%%%%%%%%%%%%%%%%%%%%%%%%%%

\begin{frame}[fragile,label=uebung30]
\frametitle{Praxis (3)}
\begin{columns}
   \column{1.1\textwidth}
\begin{enumerate}
 \setcounter{enumi}{\value{taskcounter}}

  \item Irrelevante oder geheime Dateien ignorieren
  \begin{enumerate}[a)]
   \item \verb|git status| $\rightarrow$ Hilfsdateien oder geheime Dateien
       bemerken (\verb|out.log| oder \verb|id_rsa|)
   \item[] {\scriptsize Häufig gibt es lokale Dateien, deren Änderungen nicht
           von git verfolgt werden sollen
   $\rightarrow$ Datei \verb|.gitignore| hilft}
   \item ignore-Datei ergänzen: \verb|echo "out.log" >> .gitignore|
   \item Staus-Änderung zur Kenntnis nehmen: \verb|git status|
   \item Committen: \verb|git commit -am "ignore log file"|
%    \begin{itemize}
%     \item \verb|git add .gitignore|
%     \item
%    \end{itemize}
   \item Zur Kenntnis nehmen: \verb|git status| $\rightarrow$ "`working directory clean"'
  \end{enumerate}
  \item[] Hinweis: Inhalt der Datei \verb|.gitignore| (nachprüfen):
  {\scriptsize
  \begin{verbatim}
   # This file specifies which files should not be tracked by git
   out.log
  \end{verbatim}
  }
  \vspace{-3mm}
  \begin{itemize}
  \item Bedeutung der Zeilen
  \begin{enumerate}
   \item Erklärender Kommentar (stand vorher schon drin)
   \item Ignoriere Dateien mit dem Namen \texttt{out.log}
  \end{enumerate}

  \end{itemize}
 \setcounter{taskcounter}{\value{enumi}}
\end{enumerate}
  \column{.01\textwidth}
  ~
\end{columns}
\end{frame}
%%%%%%%%%%%%%%%%%%%%%%%%%%%%%%%%%%%%%%%%%%%%%%%%%%%%%%%%%%%%%%%%%%%%%%%%%%%%%%%%

\begin{frame}[fragile,label=uebung40]
\frametitle{Praxis (3)}
\begin{enumerate}
 \setcounter{enumi}{\value{taskcounter}}
 \item Branch \textit{rom} in \textit{master} mergen
 \begin{enumerate}[a)]
 \item merge durchführen $\rightarrow$ Konflikt zur Kenntnis nehmen
 \item Überblick verschaffen: \verb|git status|  \verb|gitk --all|
 \item Manuell Konflikt in \verb|AUTHORS.md| beheben
 \item Merge abschließen durch commiten der Änderungen:\\
 {\tiny \verb|git commit --add -m "merge branch rom after manual conflict resolution"|}
 \end{enumerate}

 \setcounter{taskcounter}{\value{enumi}}
\end{enumerate}

Weitere Ideen:
\begin{itemize}
 \item Eigenen Branch anlegen mit bestimmten Eltern-Knoten
 \begin{itemize}
  \item Commit-ID herausfinden: \verb|git log| (ersten 4 Zeichen reichen)
  \item \verb|git checkout <id>|
  \item \verb|git checkout -b mybranch|
 \end{itemize}
 \item Rebase aller Branches, damit das repo linear wird
 \begin{itemize}
  \item Hintergrundwissen:
  \begin{itemize}
   \item  \verb|git help rebase|

   % das item soll auf eine Zeile passen, deshalb URL abkürzen
\item \href{https://onlywei.github.io/explain-git-with-d3/#rebase}{https://onlywei.github.io/...}
  \item dort \verb|git rebase master| eintippen, Animation anschauen und Text lesen
  \end{itemize}
 \end{itemize}

\end{itemize}

\end{frame}

%%%%%%%%%%%%%%%%%%%%%%%%%%%%%%%%%%%%%%%%%%%%%%%%%%%%%%%%%%%%%%%%%%%%%%%%%%%%%%%%
\section{Schlussbemerkungen}

\begin{frame}[fragile,label=schluss10]
\frametitle{Schlussbemerkungen (1)}
\begin{itemize}
\item github $\neq$ git
\begin{itemize}
 \item git: Freies Tool zur Versionsverwaltung
 \item github: Kommerzieller Webservice \textit{basierend auf} git
\end{itemize}
\smallskip
\pause
\item git nicht gut für (große) Binärdateien
\begin{itemize}
 \item Merges werden ungemütlich %(Binärdateien verwenden z.\,B. oft Offsets)
 \item Grund: Delta-Kompression basiert auf zeilenweisen Diffs
 \item[$\rightarrow$] \verb|.git|-Verzeichnis wird ggf. sehr groß
\end{itemize}
\smallskip
\pause
\item Nicht behandelte wichtige Konzepte/Kommandos
\begin{itemize}
 \item \verb|git fetch|, \verb|git pull|, \verb|git push|,  \verb|git rebase|, ...
 \item Siehe Cheat-Sheet
\end{itemize}
\smallskip
\pause
\item Weitere Tipps:
\begin{itemize}
\item \href{https://git-scm.com/docs}{Doku} kennen
\item \href{https://github.com/magicmonty/bash-git-prompt}{Status-Infos im Bash-Prompt}
% git-prompt-snapshot1
\item \href{https://git-scm.com/book/en/v2/Git-Basics-Git-Aliases}{Aliase} in \verb|.gitconfig| (z.B.: \verb|git co| $\rightarrow$ \verb| git checkout|)
\item \href{https://stackoverflow.com/a/7335487/333403}{Globale gitignore-Datei} anlegen
\item \href{http://nvie.com/posts/a-successful-git-branching-model/}{Bewährtes Branching-Modell} anwenden
\end{itemize}

\end{itemize}
\begin{textblock*}{50mm}[0.,0.](70mm,68mm)
\visible<4->{
\includegraphics[width=60mm]{img-src/git-prompt-snapshot1}
}
\end{textblock*}

\end{frame}

%%%%%%%%%%%%%%%%%%%%%%%%%%%%%%%%%%%%%%%%%%%%%%%%%%%%%%%%%%%%%%%%%%%%%%%%%%%%%%%%
\begin{frame}[label=schluss20]{Schlussbemerkungen (2): Blick übern Tellerrand}

    \begin{columns}[T]
        \begin{column}{.45\textwidth}

\begin{block}{Bits und Bäume}
   \begin{itemize}
    \item Digitalisierung und Nachhaltigkeit verbinden
    \item Ressourcen, Freie Software, Privatsphäre, Demokratie, ...
    \item \url{https://dresden.bits-und-baeume.org/}
   \end{itemize}
\end{block}

        \end{column}
        \begin{column}{.45\textwidth}

    \begin{block}{TU--Umweltinitiative}
        \begin{itemize}
            \item Umweltringvorlesungen ab tbd
            \item Offenes Grillen/Essen: tbd
            \item Wandern: tbd
            \item \url{https://tuuwi.de/}
        \end{itemize}
    \end{block}
    \begin{block}{Students4Future}
        \begin{itemize}
            \item $\rightarrow$ \url{https://fffdd.de/}
        \end{itemize}
    \end{block}
        \end{column}
    \end{columns}

\end{frame}

%%%%%%%%%%%%%%%%%%%%%%%%%%%%%%%%%%%%%%%%%%%%%%%%%%%%%%%%%%%%%%%%%%%%%%%%%%%%%%%%
\begin{frame}[label=uj]{Schlussbemerkungen (3)}
\begin{itemize}
 \item Fragen?\\[10mm]
 \item Unterstützung: {\tiny (im Rahmen unserer Möglichkeiten)}
   \begin{itemize}
   \item \url{https://fsfw-dresden.de/sprechstunde}
   \item \url{https://fsfw-dresden.de/git-ws}
   \item \url{https://ifsr.de/}
   \bigskip
   \item {\Large \url{kontakt@fsfw-dresden.de}}
   \smallskip
   \item {\Large \url{fsr@ifsr.de}}
   \end{itemize}
\end{itemize}

\end{frame}

%%%%%%%%%%%%%%%%%%%%%%%%%%%%%%%%%%%%%%%%%%%%%%%%%%%%%%%%%%%%%%%%%%%%%%%%%%%%%%%%

\begin{frame}[label=link10]{Quellen und Links (Auswahl)}
\tiny
\begin{itemize}
 \item \url{https://git-scm.com/documentation}
 \item \url{https://git-scm.com/documentation/external-links}
 \item \url{https://stackoverflow.com/questions/tagged/git}
% \url{https://learn.adafruit.com/an-introduction-to-collaborating-with-version-control/initializing-a-repository-and-making-commits}
\item ...
\end{itemize}
\end{frame}

%%%%%%%%%%%%%%%%%%%%%%%%%%%%%%%%%%%%%%%%%%%%%%%%%%%%%%%%%%%%%%%%%%%%%%%%%%%%%%%%


% \begin{frame}[label=schluss0]
% \frametitle{Noch fehlende Inhalte (kurz oder mit eigener Folie)}
% \begin{itemize}
% \item diff \checkmark / difftool
% \item branches \checkmark
% \item Visualisierung eines repos (Graph mit Pfeilen) \checkmark
% \item HEAD\checkmark , hash bzw. ref \checkmark
% \item merge (Konflikte) \checkmark, fast-forward \checkmark
% \item .gitignore \checkmark
% \item github $\neq$ git \checkmark
% \item git nicht gut für Binärdateien \checkmark
% \item bash extension
% \item aliase in .gitconfig \checkmark
% \item erwähnen: push,  rebase, \checkmark
% \item link auf branching model \checkmark
% \item globale gitignore \checkmark
% \item Werbung am Anfang kürzen und aktualisieren
%
% \end{itemize}
% \end{frame}



\end{document}
